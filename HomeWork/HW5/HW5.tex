\documentclass{article}
\usepackage[UTF8]{ctex}
\usepackage[left=2cm,right=2cm,top=1.5cm,bottom=1.5cm]{geometry}
\usepackage{listings}
\usepackage{xcolor}
\usepackage{fontspec}
\usepackage{amsmath}
\usepackage{tikz}
\usepackage[thmmarks,amsmath]{ntheorem}
\setmonofont{Consolas}
\lstset{numbers=left, numberstyle=\tiny, keywordstyle=\color{blue!70}, commentstyle=\color{red!50!green!50!blue!50}, rulesepcolor=\color{red!20!green!20!blue!20}, frame=shadowbox, basicstyle=\ttfamily, breaklines=true, tabsize=4}

\begin{document}
	\title{HW 5}
	\author{肖桐 PB18000037}
	\date{2020 年 11 月 17 日}
	\maketitle

	\newtheorem*{solution}{解}

	\begin{solution}\textnormal{\textbf{1.}}
		{\rm
		\begin{lstlisting}[language=C]
S *MAKESET(int x)
{
	pSet = (S *)malloc(sizeof(S));	//pSet 指向一个新的集合头节点
	pNew = (Node *)malloc(sizeof(Node));	//pNew 指向一个新的数据节点
	pNew->Value = x;
	pNew->pNext = NULL;
	pNew->Set = pSet;	//指向头节点
	pSet->head = pNew;
	pSet->tail = pNew;
	pSet->Length = 1;
	return pSet;
}

S *FIND_SET(Node *x)
{
	return x->Set;
}

S *UNION(Node *p1, Node *p2)
{
	S *pSet1 = FIND_SET(p1);
	S *pSet2 = FIND_SET(p2);
	if (pSet2->Length > pSet1->Length)
	{
		Node *z = pSet1->head;
		while (z != NULL)
		{
			z->Set = pSet2;
			z = z->pNext;
		}
		pSet1->tail->pNext = pSet2->head;
		pSet2->head = pSet1->head;
		pSet2->Length += pSet1->Length;
		free(pSet1);
		return pSet2;
	}
	else
	{
		Node *z = pSet2->head;
		while (z != NULL)
		{
			z->Set = pSet1;
			z = z->pNext;
		}
		pSet2->tail->pNext = pSet1->head;
		pSet1->head = pSet2->head;
		pSet1->Length += pSet2->Length;
		free(pSet2);
		return pSet1;
	}
}
		\end{lstlisting}
		}
	\end{solution}
	\begin{solution}\textnormal{\textbf{2.}}
		设$Value[i, j]$为选择前$i$个物品中的其中几个加入背包中, 且使得总重量不超过$j$所取得的最大价值. 初始化$Value[n + 1, W + 1]$\newline
		设$n$件物品重量为$w_1, w_2, \dots, w_n$, 对应的价值分别为$v_1, v_2, \dots, v_n$.\newline
		则有最优子结构:$Value[i, j] = \max\{Value[i - 1, j], Value[i - 1, j - w_i] + v_i\}$.\newline
		理解为每次尝试将最新纳入考虑的第$i$件物品装入背包, 若将第$i$件物品装入背包能够使得总价值增加的话, 那就将第$i$件物品装入背包, 否则保持原来的状态.\newline
		因此采取自底向上的方法可以写出伪代码:
		{\rm
		\begin{lstlisting}[language=C]
for (int i = 0; i < n + 1; i++)
{//初始化
	Value[i, 0] = 0;
}
for (int j = 0; j < W + 1; j++)
{//初始化
	Value[0, j] = 0;
}

for (int i = 1; i < n + 1; i++)
{
	for (int j = 1; j < W + 1; j++)
	{
		Value[i, j] = max{Value[i - 1, j], Value[i - 1, j - w_i] + v_i};
	}
}
		\end{lstlisting}
		}
		因为内循环复杂度为$O(nW)$, 因此算法总复杂度为$O(nW)$.
	\end{solution}
\end{document}