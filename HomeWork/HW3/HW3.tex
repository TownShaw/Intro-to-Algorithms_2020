\documentclass{article}
\usepackage[UTF8]{ctex}
\usepackage[left=2cm,right=2cm,top=1.5cm,bottom=1.5cm]{geometry}
\usepackage{listings}
\usepackage{xcolor}
\usepackage{fontspec}
\usepackage{amsmath}
\usepackage{tikz}
\usetikzlibrary{calc}
\usepackage[thmmarks,amsmath]{ntheorem}
\setmonofont{Consolas}

\begin{document}
	\title{HW3}
	\author{肖桐 PB18000037}
	\date{2020 年 10 月 23 日}
	\maketitle

	\newtheorem*{solution}{解}

	\begin{solution}\textnormal{\textbf{1.}}
		稳定排序算法:插入排序、归并排序.\newline
		非稳定排序算法:堆排序、快速排序、计数排序.\newline
		改进算法:可以将每个数据都包装为一个结构体, 其中不仅包含原来的数据$value$,
		还应该包含每个数据在原来数组中的下标$index$.\newline
		这样只需要改变在排序时的判断条件:$(value_1 < value_2)\ ||\ ((value_1 == value_2)\ \&\&\ (index_1 < index_2))$才能进行交换.\newline
		额外时间开销为每次比较都要额外比较两个表达式, 额外空间开销为原来的一倍.
	\end{solution}
	\begin{solution}\textnormal{\textbf{2.}}
		划分时每次取当前数组中最大的元素可以使得发生最坏情况. 因此划分序列为$9, 8, 7, 6, 5, 4, 3, 2, 1, 0$.
	\end{solution}
	\begin{solution}\textnormal{\textbf{3.}}
		先考虑构造二叉树的最佳情况. 即是每次向二叉树中插入节点时二叉树深度都为$\lfloor\lg n\rfloor + 1$.
		则时间复杂度为:
		$$
		\begin{aligned}
			\begin{split}
				\sum_{k = 1}^{n}(\lfloor\lg k\rfloor + 1) &\leq \sum_{k = 1}^{n}(\lg k + 2) \\
				&= 2n + \sum_{k = 1}^{n}\lg k = 2n + \lg\prod_{k = 1}^{n}k \\
				&\leq 2n + \lg n^n = 2n + n\lg n \\
				&= \Omega(n\lg n)
			\end{split}
		\end{aligned}
		$$
		又因为最坏情况下的时间复杂度必然比最佳情况的要大. 因此最坏情况下时间复杂度也是$\Omega(n\lg n)$.
	\end{solution}
	\begin{solution}\textnormal{\textbf{4.}}
		记初始节点为$x$, $x$经$k$次\textnormal{TREE\_SUCCESSOR}操作后到达的节点, $z$为$x, y$高度最高的公共祖先节点.\newline
		则因为\textnormal{TREE\_SUCCESSOR}操作实际上是一部分的树的遍历操作, 因此每条树边不会被访问两次以上, 每个节点不会被访问$3$次以上.\newline
		同时, 路径$x \to z$和$y \to z$上值不在$x, y$之间的节点被最多访问一次. 又因为每次\textnormal{TREE\_SUCCESSOR}操作访问节点个数的上界为$h$.\newline
		因此总的访问节点个数的上界为$3k + 2h = O(k + h)$. 因此时间复杂度也为$O(k + h)$.
	\end{solution}

	$$(N, C_{\text{in}}, H, W)$$
	$$(N, C_{\text{out}}, H_{\text{out}}, W_{\text{out}})$$
	$$\text{out}(N_i, C_{\text{out}_j}) = \text{bias}(C_{\text{out}_j}) +  
    \sum_{k = 0}^{C_{\text{in}} - 1} \text{weight}(C_{\text{out}_j}, k) \star \text{input}(N_i, k)$$
\end{document}